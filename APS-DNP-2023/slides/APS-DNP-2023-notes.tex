\documentclass[8pt, xcolor={dvipsnames}, sans, mathserif]{beamer}

\usepackage{fontspec}
\usepackage{fontawesome5}
\usepackage{mathrsfs}
\usepackage{amsmath, amssymb}
\usepackage{braket}
\usepackage{graphicx}
\usepackage{hyperref}
\usepackage[absolute,overlay]{textpos}
\usepackage[font=tiny, skip=0.5pt]{caption}

\captionsetup[figure]{labelformat=empty}

% Define the custom theme
\mode<presentation>

\setbeamertemplate{footline}[frame number]
\setbeamertemplate{caption}[default]
\setbeamertemplate{navigation symbols}{}

\setbeamerfont{footnote}{size=\tiny}


% Some custom commands
\newenvironment{List}[2]
{\begin{textblock}{#1}#2
\begin{itemize}}
{\end{itemize}
\end{textblock}}

\newenvironment{Pic}[2]
{\begin{textblock}{#1}#2
\begin{figure}}
{\end{figure}
\end{textblock}}

\begin{document}


\begin{frame}
\begin{List}{15.}{(0.1, 0.1)}

  \item Proton spin 1/2 fermion made with 3 quarks: $uud$.

  \item According to ``Ellis-Jaffe" sum rule : quarks contributes about 60\% of the total spin of the proton, but EMC found about that the 20\% of the total spin of the proton is contributed by the quarks.

  \item According to ``Jaffe-Manohar" spin decomposition, 25\% of the proton spin comes from intrinsic spin of the quarks, about 50\% comes from the angular momentum of the quarks, $g$ contributes to the about 30\% of the total angular momentum of the proton.

  \item  TMDs : distributions of the hadron's quark or gluon momenta that are perpendicular to the momentum transfer between the beam and the hadron.

  \item Provide information on the confined motion of quarks and gluons inside the hadron and complement the information on the hadron structure.

  \item Depending on the polarization of the quark/nucleon polarization, we have 6 TMDs. Polarized nucleon and un-polarized quark - Sivers function. Un-polarized nucleon and polarized quarks - Boer-Mulders function.

\end{List}
\end{frame}

\begin{frame}
\begin{List}{15.}{(0.5, 0.5)}

  \item BMF : Describes the net polarization of quarks inside an unpolarized proton.

  \item $h_{1}^{\perp}$ $\rightarrow$ quark distribution that quantifies a particular spin-orbit correlation.

  \item Useful for probing the internal structure of the proton.

  \item

\end{List}
\end{frame}

\end{document}
