\documentclass{article}

\title{Utilizing Deep Neural Networks for the Extraction of Drell-Yan Angular Coefficients in $pp$ Collisions with a 120 GeV Beam Energy}

\author{Dinupa Nawarathne, for the SeaQuest Collaboration}

\begin{document}

\maketitle

\begin{abstract}
Understanding the total spin of protons plays a major role in comprehending the proton's structure. The $\cos2\phi$ asymmetry
in the Drell-Yan process, where $\phi$ denotes the azimuthal angle of the $\mu^{+}\mu^{-}$ pair in the Collins-Soper
frame, can be described by the Boer-Mulders (BM) function. This function characterizes the transverse-polarization asymmetry
of quarks within an unpolarized hadron and arises from the coupling between the quark's transverse momentum and transverse
spin inside the hadron. SeaQuest is a fixed-target Drell-Yan experiment conducted at Fermilab, which involves an unpolarized
proton beam colliding with an unpolarized LH2 and LD2 targets. The $\cos2\phi$ asymmetry is determined by detecting $\mu^{+}\mu^{-}$
pairs. Accurately extracting the $\cos2\phi$ asymmetry is important for determining the BM function. Measurements obtained
from experiments typically require correction for detector inefficiencies, smearing, and acceptance. Traditionally, these
corrections involve ``unfolding" the detector-level measurements through matrix operations. However, in higher dimensions
in parameter space, these conventional methods fail to scale effectively. To overcome these limitations, we propose a
novel approach that utilizes Deep Neural Networks for directly extracting the angular coefficients using high-dimensional
information from the detector level. Neural networks excel in approximating nonlinear functions, making them suitable for
representing the full phase space for parameter optimization. In this presentation, we will explain the design of the neural
network architecture, training strategies, and outline our plans to achieve conclusive results.
\end{abstract}

\end{document}

