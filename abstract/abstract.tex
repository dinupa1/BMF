\documentclass{article}

\title{Extraction of Drell-Yan Angular Coefficients using Neural Network-based Classifiers}

\author{Dinupa Nawarathne}

\begin{document}

\maketitle


% original
% \begin{abstract}
%
% The Drell-Yan dilepton angular distribution is a valuable tool for unraveling the structure of hadrons.
% For example, the $\cos2\phi$ dependence of the angular distribution can be used to extract the Boer-Mulders (BM) function,
% which characterizes the net polarization of quarks within an unpolarized proton.
% The BM function captures the presence of a handedness phenomenon within the proton and represents a quark distribution
% that quantifies a specific spin-orbit correlation. Conventional methods for extracting the angular coefficients typically
% involve unfolding low-dimensional detector data, which may not fully exploit the complete phase space for optimal
% parameter optimization. To overcome this limitation, we propose a novel approach utilizing Neural Network-based Classifiers to
% directly extract the angular coefficients using high-dimensional information from the detector level. In this presentation,
% we will explain the design of the neural network architecture, training strategies, and outline our plans to achieve conclusive results.
%
% \end{abstract}

\begin{abstract}
Study of angular distributions in the Drell-Yan process is a valuable tool for unraveling the structure of hadrons.
Measuring the $\cos2\phi$ angular dependence, where $\phi$ denotes the azimuthal angle of dimuons in the Collins-Soper
frame, can be used to extract the Boer-Mulders (BM) function. The BM function describes the transverse-polarization
asymmetry of quarks within an unpolarized hadron and is a result of the coupling between transverse momentum and
transverse spin of the quarks inside the hadron. Conventional methods for extracting the angular-distribution coefficients
typically involve unfolding low-dimensional detector data, which may not fully exploit the complete phase space for
best parameter optimization. To overcome this limitation, we propose a novel approach utilizing Neural Network-based
Classifiers to directly extract the angular coefficients using high-dimensional information at the detector level.
In this presentation, we will explain the design of the neural network architecture, training strategies, and outline
our plans to achieve conclusive results.
\end{abstract}

\end{document}