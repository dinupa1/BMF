% dinupa3@gmailcom
% 13-feb-2023


\documentclass[10pt, xcolor={dvipsnames}, aspectratio = 169, sans,mathserif]{beamer}

\usepackage{fontspec}
\usepackage{fontawesome5}
\usepackage{mathrsfs}
\usepackage{amsmath, amssymb}
\usepackage{graphicx}
\usepackage{hyperref}
\usepackage{physics}
\usepackage[absolute,overlay]{textpos}
\usepackage[font=tiny]{caption}

\mode<presentation>
{
\usefonttheme{serif}
\setmainfont{JetBrains Mono}
\definecolor{nmsured}{RGB}{137,18,22} % custom colors
\setbeamercolor{title}{bg=White,fg=nmsured}
\setbeamercolor{frametitle}{bg=White,fg=nmsured}
\setbeamercolor{section number projected}{bg=nmsured,fg=White}
\setbeamercolor{subsection number projected}{bg=nmsured,fg=White}
\setbeamertemplate{items}{\color{nmsured}{\faAngleDoubleRight}}
\setbeamertemplate{section in toc}[square]
\setbeamertemplate{subsection in toc}[square]
\setbeamertemplate{footline}[frame number]
\setbeamertemplate{caption}[numbered]
\setbeamerfont{footnote}{size=\tiny}
\setbeamercovered{invisible}
\usefonttheme{professionalfonts}
%\setbeamertemplate{background}[grid][color=nmsured!15] % set background
\setbeamertemplate{navigation symbols}{} % remove navigation buttons
}


\title{Boer Mulders Functions}
\subtitle{Notes}
\author{Dinupa}

%%%%%%%%%%%%%%%%%%%%%%%%%%%%%%%%%%%%%%%%%%%%%%%%%%%%%%%%%%%%%
% some custom commands
\newcommand{\MyBox}[4]
{
\begin{textblock}{#1}(#2, #3)
#4
\end{textblock}
}

\newcommand{\MyPic}[2]
{
\begin{figure}[fragile]
    \centering
    \includegraphics[width=#1cm]{../imgs/#2.png}
\end{figure}
}

\newcommand{\MySlide}[1]
{
\begin{frame}
#1
\end{frame}
}

\newcommand{\MyMath}[1]
{
\begin{equation*}
#1
\end{equation*}
}

\newcommand{\MyList}[1]
{
\begin{itemize}
#1
\end{itemize}
}
%%%%%%%%%%%%%%%%%%%%%%%%%%%%%%%%%%%%%%%%%%%%%%%%%%%%%%%%%%%%%%%%%%%%


\begin{document}

\begin{frame}
    \maketitle
\end{frame}

\MySlide{
\frametitle{Drell-Yan Process}

\MyBox{10.0}{0.5}{2.0}{

\MyList{

    \item DY angular distribution;

    \MyMath{
    \frac{d\sigma}{d\Omega} \propto 1 + \lambda cos^{2}\theta + \mu sin2\theta cos\phi + \frac{\nu}{2} sin^{2}\theta cos2\phi
    }

    \item In "naive" DY model;

    \MyList{

    \item Transverse momentum of the quark is ignored.

    \item No gluon emission is considered.

    \item $\lambda = 1$, $\mu = \nu = 0$.
    }

    \item QCD effects and non-zero intrinsic transverse momentum of the quarks;

    \MyList{

    \item $\lambda \neq 1$, $\nu, \mu \neq 0$.

    \item Satisfy Lam-Tung relation;

    \MyMath{
    1-\lambda = 2 \nu
    }
    }
}
}
\MyBox{6.0}{10.0}{1.0}{\MyPic{6.0}{DY_ang_coef}}
}

\MySlide{
\frametitle{Drell-Yan Process}

\MyList{
\item E866 and E615 results on DY angular distribution strongly suggest that new effects beyond conventional perturbative QCD are present.

\item Boer-Mulders function can explain this results;

\MyList{
\item $k_{T}$ dependent parton distribution function $h_{1}^{\perp}$.

\item Characterizes the correlation of a quark’s transverse spin and its transverse momentum $k_{T}$ in an unpolarized nucleon.

\item E866 experiment;

\MyList{
\item 1st report on the azimuthal angular distributions for proton-induced DY.

\item  Proton-induced Drell-Yan data provide a stringent test of theoretical models.

\item $\cos 2\phi$ dependence is expected to be much reduced in proton-induced DY if the underlying mechanism involves the Boer-Mulders functions; due to the expectation that the BM functions are small for the sea-quark.

\item However, if the QCD vacuum effect is the origin of the $\cos 2\phi$ angular dependence, then the azimuthal behavior of proton-induced DY should be similar to that of pion-induced DY.

\item 1st measurement of the validity of the LT relation for proton-induced DY.
}
}
}

\tiny{
source :
\href{https://arxiv.org/abs/hep-ex/0609005}{arXiv:hep-ex/0609005}
}
}

\MySlide{
\frametitle{Drell-Yan Process}

\MyBox{9.0}{0.5}{1.5}{
\MyList{

\item For NA10 data, Boer assumed that $h_{1}^{\perp}$ proportional to the spin-averaged parton distribution function $f_{1}$.

\MyMath{
h_{1}^{\perp}(x, k_{T}^{2}) = C_{H} \frac{\alpha_{T}}{\pi} \frac{M_{C}M_{H}}{k_{T}^{2} + M_{C}^{2}} e^{\alpha_{T}k_{T}^{2}} f_{1}(x)
}

\item The $\cos 2 \phi$ dependence then results from the convolution of the pion $h_{1}^{\perp}/f_{1}$ term with the nucleon $h_{1}^{\perp}/f_{1}$ term, and the parameter $\nu$;

\MyMath{
\nu = 16 k_{1} \frac{p_{T}^{2} M_{C}^{2}}{(p_{T}^{2} + 4M_{C}^{2})^{2}}
}

Predict large $\cos 2 \phi$ dependence originating from QCD vacuum effects, suggest that $h_{1}^{\perp}$ for sea quarks are significantly smaller than those for valence quarks.
}
}

\MyBox{6.0}{10.0}{2.0}{\MyPic{6.0}{nu_E866}}
}

\MySlide{
\frametitle{Boer-Mulders Function ($pD$)}

\MyList{

\item Definition;

\MyMath{
h_{1}^{\perp}(x, k_{T}^{2})\epsilon^{ij}_{T}k_{T_{j}} = \frac{M}{2}F.T.\bra{P}\bar{\psi}(0)\mathcal{L}_{C}(0, \epsilon) \gamma^{i}\gamma^{+}\gamma_{5}\psi(\epsilon)\ket{P}|_{\epsilon^{+} = 0}
}

\item Describes the net polarization of quarks inside an unpolarized proton.

\item If this function is nonzero, then it reflects the presence of a handedness inside the proton $\vec{P} \cdot (\vec{k}_{T} \times \vec{s}_{T})$.

\item For DY process;

\MyMath{
\nu_{pD} = \frac{2\mathcal{F}[\chi(e^{2}_{u}h^{\perp, u}_{1} + e^{2}_{d}h_{1}^{\perp, d})(h_{1}^{\perp, \bar{u}} + h_{1}^{\perp, \bar{d}})] + q \leftrightarrow \bar{q}}{\mathcal{F}[(e^{2}_{u}f^{u}_{1} + e^{2}_{d}f^{d}_{1})(f_{1}^{\bar{u}} + f_{1}^{\bar{d}})]+ q \leftrightarrow \bar{q}}
}
}
}

\MySlide{
\frametitle{Boer-Mulders Function ($pD$)}

\MyList{

\item Parametrize $h_{1}^{\perp}(x, \vec{p}_{T}^{2})$ in the factorized form as;

\MyMath{
h_{1}^{\perp, q} (x, \vec{p}_{T}) = h_{1}^{\perp, q}(x)\frac{exp(-\vec{p}_{T}^{2})/p_{bm}^{2}}{\pi p^{2}_{bm}}
}

\item $h_{1}^{\perp, q}$ can be written as;

\MyMath{h_{1}^{\perp, q} (x) = H_{q} x^{c} (1-x) f_{1}^{q} (x)}

\item TMD unpolarized distribution function

\MyMath{f_{1}^{q}(x, \vec{p}_{T}^{2}) = f_{1}^{q}(x)\frac{exp(-\vec{p}_{T}^{2}/p^{2}_{unp})}{\pi p^{2}_{unp}}}
}
}


\MySlide{
\frametitle{Boer-Mulders Function ($pD$)}

\MyList{
\item Integrating over $x_{1}$ and $x_{2}$;

\MyMath{
\nu_{pD}(Q_{T}) = \frac{p^{2}_{unp}}{2M^{2}}\frac{\int dx_{1}\int dx_{2} [\alpha]Q_{T}^{2}exp(-Q_{T}^{2}/2p_{bm}^{2})}{p^{2}_{bm} \int dx_{1} \int dx_{2}[\beta] exp(-Q_{T}^{2}/2p_{unp}^{2})}
}

where $x_{1}x_{2}s = Q^{2} + Q_{T}^{2}$.

\item This expression is fitted to the E866 data.

\MyPic{7.0}{fit_coef}
}
}

\MySlide{
\frametitle{Boer-Mulders Function ($pD$)}
\MyPic{14.0}{train_fit}
}

\MySlide{
\frametitle{Boer-Mulders Function ($pD$)}
\MyPic{14.0}{test_fit}
}

\MySlide{
\frametitle{Boer-Mulders Function ($pp$)}

\MyBox{15.0}{0.5}{1.5}{\MyList{\item We can get similar experssion for $pp$ DY;}}
\MyBox{7.0}{0.5}{3.0}{\MyPic{7.0}{pp_eq}}
\MyBox{7.0}{8.0}{3.0}{\MyPic{7.0}{pp_plot}}
}

\MySlide{
\frametitle{Boer-Mulders Function ($p\vec{p}$)}

\MyBox{15.0}{0.5}{1.5}{
\MyList{
\item To seperate $H_{q}$ (valance quarks) and $H_{\bar{q}}$ (sea quark) we can introduce free coef. $\omega$;
}
}

\MyBox{7.0}{0.5}{3.0}{\MyPic{7.0}{free_coef}}
\MyBox{7.0}{8.0}{3.0}{\MyPic{7.0}{free_coef_eq}}
}

\MySlide{
\frametitle{Boer-Mulders Function ($p\vec{p}$)}

\MyBox{7.0}{0.5}{2.0}{\MyPic{7.0}{free_coef_plot}}

\MyBox{7.0}{8.0}{2.0}{
\tiny{
source:
\href{https://arxiv.org/abs/0803.1692}{arXiv:0803.1692 [hep-ph]},
\href{https://arxiv.org/abs/hep-ph/9711485}{arXiv:hep-ph/9711485}
}
}
}

\MySlide{
\frametitle{Pipeline}

\MyBox{15.0}{0.5}{2.0}{
\MyList{
\item Start with MC data;

\MyList{
\item Get $\phi$ vs. $\cos\theta$ distributions in $p_{T}$, $x_{1}$, $x_{2}$, $x_{F}$, $m_{\mu\mu}$ bins.

\item Use unfolding to make the corrections.

\item Fit the $\phi$ vs. $\cos\theta$ ditributions to extract $\lambda$, $\mu$, $\nu$.

\item Use $\nu$ values to extract BM function.

\item Cross-check (probably with Kei)
}

\item Real data;

\MyList{
\item Understand the reco. cuts.

\item Background subtraction (Combinatorial Background, \href{https://github.com/abinashpun/e906-root-ana}{e906-root-ana})

\item Use the same steps as in MC data.
}
}
}
}


\end{document}
